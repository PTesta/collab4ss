%        File: collab4ss.tex
%     Created: Tue Feb 22 02:00 PM 2011 C
% Last Change: Tue Feb 22 02:00 PM 2011 C
%
\documentclass[]{article}

\usepackage{hyperref}

\author{Mark Fredrickson \and Paul F. Testa} % \and you!
\title{Collaboration for Social Scientists, or Software is the Easy Part}
\begin{document}

\maketitle
\section{Collaboration Basics}

% text files/documents of simple structure
% sideband communication (``Hey! I'm working on that!'')
% place in context of other articles in TPM

In this article, we consider two different modes of collaboration: synchronous
and asynchronous. When working synchronously, contributors are both working on
\emph{the same portions of the research at the same time}. We provide some
suggestions for maximizing time spent working together. Of course, virtually
any research project will require collaborators to spend time working on
either different portions of the project or working on the same sections but
at different times. We label this form of collaboration asynchronous.
Asynchronous collaboration requires more careful attention to dividing labor,
and we spend more time providing software solutions in this domain.

\section{Synchronous Collaboration}

While it might appear collaborators at the same institution or who can
frequently meet face to face will benefit from synchronous collaboration
techniques, many of these techniques rely on networked computers or can be
applied over video chat or speaker phone.

We begin by importing some techniques from software engineering. In recent
years, so-called ``Agile'' programming and project management aproaches to
software engineering have become popular, especially at start ups and younger
development shops. Many techniques fall under the umbrella of ``Agile''
methods, including suggestions for organizing teams, minimizing unnecessary
meetings, and communicating frequently changing client requests. While social
scientists could benefit from these suggestions, we tend to be our own clients
and work in smaller teams than programmers. One technique we do think would be
of benefit to social scientists would be the concept of \emph{Pair
Programming,} sometimes also called ``eXtreme Programming'' (XP). Pair
programming places two programmers at the same computer: one screen, one
keyboard, two heads. One programmer takes the lead to write software, while
the second provides suggestions, acts as a sounding board, catches errors, and
questions assumptions made by the first programmer. 

While it may sound wasteful to place two collaborators in front of a single
computer and have them both focus on the same task, the technique can lead to
\emph{more} code being written and \emph{higher quality} software as well. The
key insight is that typing is rarely the bottleneck for producing code. Having
a second person on hand to help with the concepts, design, and implementation
cuts down on time spent chasing dead-ends or time wasted on simple bugs.
If you have spent several hours on a problem only to realize your mistake
while explaining the problem to someone, you will see the immediate benefits
of pair programming. 

In practice, pair programming need not have both subjects staring at the same
screen at the same time. One programmer may be writing code, while the other
looks into API documentation, writes unit tests, or provides documentation,
but is immediately available to support the first programmer. Additionally,
collaborators need not be in the same physical space. There are several tools
for real-time co-editing of documents. Wikipedia provides a fairly detailed
list of \href{http://en.wikipedia.org/wiki/Collaborative_real-time_editor}{collaborative real-time
editors}. All of these editors allow multiple authors to simultaneously edit
documents, which may even be a useful feature to pair programmers in the same
physical space. Since this issue of TPM is strongly encouraging learning and
using a text editor, you may wish to favor editors that allow for simultaneous
editing. At a minimum, \href{http://www.gnu.org/software/screen/}{GNU Screen}
provides an immediate solution for Emacs and VIM users who wish to pair program.
More advance uses may require a editor plugin or separate editor.

In addition to managing file editing, some real time editors also facilitate
verbal communication. Of course, if your editor does not immediately provide
this service, a call via \href{http://www.skype.com}{Skype} or
\href{http://chat.google.com}{Google Chat} can fulfill communication needs. Of
course, these tools can also be of use to collaborators, even they forgo the
pair programming model.
% pair programming
% other agile methods
% screen & google docs
% irc/aim/jabber

\section{Asynchronous Collaboration}
% dropbox/shared files
% wiki
% version control
% make files to explicitly state dependences

\section{Further Reading}

% TODO appendix and a bib file
\end{document}


